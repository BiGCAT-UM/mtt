\documentclass[a5paper,9pt]{extarticle}
\usepackage{palatino}
\usepackage[a5paper,left=1.5cm,right=1.5cm,top=1cm,bottom=0cm]{geometry}
\geometry{twoside=false}
\usepackage{enumitem}

\author{\large{Robbert Harms, Maastricht, 1 oktober 2017}}
\title{
	\large{Propositions accompanying the dissertation}\\ \vspace{2.5mm}
	\LARGE{\textbf{The Maastricht Thesis Template}}\\
	\large{\textbf{With a subtitle}}
}
\date{}

\begin{document}

\maketitle
\thispagestyle{empty}

\begin{enumerate}[leftmargin=*]
	\begin{large}
		\item A good template makes a PhD thesis so much easier to write. \begin{flushright}-\, Robbert Harms\end{flushright}
		\item This is not a proposition.
		\begin{flushright}-\,Free after Magritte\end{flushright}
	\end{large}
\end{enumerate}
\end{document}


%A minimum of eight and a maximum of eleven propositions shall be appended to the dissertation. 
% Four of these propositions must be related to the topic of the dissertation. 
% Three other propositions must be related to the field of science of the doctoral candidate, but not the topic of the dissertation. 
% One proposition must be related to the valorisation opportunities for the topic of the dissertation. 
% Other propositions need not be related to the topic of the dissertation or the doctoral candidate’s field. 

%The supervisor needs to agree with the propositions.
